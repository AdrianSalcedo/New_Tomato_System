%% Based on a TeXnicCenter-Template by Gyorgy SZEIDL.
%%%%%%%%%%%%%%%%%%%%%%%%%%%%%%%%%%%%%%%%%%%%%%%%%%%%%%%%%%%%%

%------------------------------------------------------------
%
\documentclass{article}%
%Options -- Point size:  10pt (default), 11pt, 12pt
%        -- Paper size:  letterpaper (default), a4paper, a5paper, b5paper
%                        legalpaper, executivepaper
%        -- Orientation  (portrait is the default)
%                        landscape
%        -- Print size:  oneside (default), twoside
%        -- Quality      final(default), draft
%        -- Title page   notitlepage, titlepage(default)
%        -- Columns      onecolumn(default), twocolumn
%        -- Equation numbering (equation numbers on the right is the default)
%                        leqno
%        -- Displayed equations (centered is the default)
%                        fleqn (equations start at the same distance from the right side)
%        -- Open bibliography style (closed is the default)
%                        openbib
% For instance the command
%           \documentclass[a4paper,12pt,leqno]{article}
% ensures that the paper size is a4, the fonts are typeset at the size 12p
% and the equation numbers are on the left side
%
\usepackage{amsmath}%
\usepackage{amsfonts}%
\usepackage{amssymb}%
\usepackage{graphicx}
%-------------------------------------------
\newtheorem{theorem}{Theorem}
\newtheorem{acknowledgement}[theorem]{Acknowledgement}
\newtheorem{algorithm}[theorem]{Algorithm}
\newtheorem{axiom}[theorem]{Axiom}
\newtheorem{case}[theorem]{Case}
\newtheorem{claim}[theorem]{Claim}
\newtheorem{conclusion}[theorem]{Conclusion}
\newtheorem{condition}[theorem]{Condition}
\newtheorem{conjecture}[theorem]{Conjecture}
\newtheorem{corollary}[theorem]{Corollary}
\newtheorem{criterion}[theorem]{Criterion}
\newtheorem{definition}[theorem]{Definition}
\newtheorem{example}[theorem]{Example}
\newtheorem{exercise}[theorem]{Exercise}
\newtheorem{lemma}[theorem]{Lemma}
\newtheorem{notation}[theorem]{Notation}
\newtheorem{problem}[theorem]{Problem}
\newtheorem{proposition}[theorem]{Proposition}
\newtheorem{remark}[theorem]{Remark}
\newtheorem{solution}[theorem]{Solution}
\newtheorem{summary}[theorem]{Summary}
\newenvironment{proof}[1][Proof]{\textbf{#1.} }{\ \rule{0.5em}{0.5em}}

\begin{document}

\title{New System}
\author{Gabriel A. Salcedo Varela
\\Universidad de Sonora, M\'exico}
\date{}
\maketitle

\begin{abstract}
This is a sample document which shows the most important features of the Standard
\LaTeX\ Journal Article class.
\end{abstract}
%%%%%%%%%%%%%%%%%%%%%%%%%%%%%%%%%%%%%%%%%%%%%%%%%%%%%%%%%%%%%%%%%%%%%%%%%%%%%%%%%%%%%%%%%%%%%%%%%%%%%%%%%%%%%%%%%%%%%%%%%%%%%%%%%%%%%%%%%%%%%%%%%%%%%%%%%%%%%%%%%%%%%%%%%%%%%%%%%%%%%%%%%%%%%%%%%%%%%%%%%%%%%%%%%%%%%%%%%%%%%%%%%%%%%%%%%%%%%%%%%%%%%%%%%%%%%%%%%%%%%%%%%%%%%%%%%%%%%%%%%%%%%%%%%%%%%%%%%%%%%%%%%%%%%%%%%%%%%%%%%%%%%%%%%%%%%%%%%%%%%%%%%%%%

\section{Tomato Model}
	\begin{align}
		\dot{S^y_p} &= -\beta^y_p S^y_p I_v + r^1_y L^y_p + r^2_y I^y_p + r_a I^a_p-\alpha S^y_p\\
		\dot{S^a_p} &= -\beta^a_p S^a_p I_v + \alpha S^y_p\\
		\dot{L^y_p} &= \beta^y_p S^y_p I_v - b_y L^y_p - r^1_y L^y_p\\
		\dot{L^a_p} &= \beta^a_p S^a_p I_v -b_a L^a_p\\
		\dot{I^y_p} &= b_y L^J_p - r^2_y I^y_p\\
		\dot{I^a_p} &= b_a L^a_p - r_a I^a_p\\
		\dot{S_v} &= -\beta^y_v S_v I^y_p - \beta^a_v S_v I^a_p -\gamma S_v +(1-\theta) \mu \\
		\dot{I_v} &= \beta^y_v S_v I^y_p + \beta^a_v S_v I^a_p - \gamma I_v +\theta \mu
	\end{align}
\section{Free disease equilibrium and $R_0$}

	$x_0=(N_p,0,0,0,0,0,\frac{\mu}{\gamma},0)$.

	In this case we can cumpute the $R_0$ from the next generation matrix, then

	\begin{equation}
	F(S^y_p,S^a_p,L^y_p,L^a_p,I^y_p,I^a_p,S_v,I_v) =
	\begin{bmatrix} 0 & 0 & 0 & 0 & \beta^Y_p S^y_p \\ 0 & 0 & 0 & 0 & \beta^a_p S^a_p \\ 0 & 0 & 0 & 0 & 0 \\ 0 & 0 & 0 & 0 & 0 \\ 0 & 0 & \beta^y_v S_v & \beta^a_v S_v & 0
	\end{bmatrix}
	\end{equation}
	and 
	\begin{equation}
	V(S^y_p,S^a_p,L^y_p,L^a_p,I^y_p,I^a_p,S_v,I_v) =
	\begin{bmatrix} b_y + r^1_y & 0 & 0 & 0 & 0 \\ 0 & b_a & 0 & 0 & 0 \\ -b_y & 0 & r^2_y & 0 & 0 \\ 0 & -b_a & 0 & r_a & 0 \\ 0 & 0 & 0 & 0 & \gamma \end{bmatrix}
	\end{equation}

	And the next generation matrix is defined by $K = F(x_0)V(x_0)^{-1}$

	\begin{equation}
	K =
	\begin{bmatrix} 0 & 0 & 0 & 0 & \frac{\beta^y_p N_p}{\gamma} \\ 0 & 0 & 0 & 0 & 0 \\ 0 & 0 & 0 & 0 & 0 \\ 0 & 0 & 0 & 0 & 0 \\ \frac{\beta^y_v\mu b_y}{\gamma(b_yr^2_y+r^1_yr^2_y)} & \frac{\beta^a_v\mu }{\gamma b_a} & \frac{\beta^y_v\mu }{\gamma r^2_y} & \frac{\beta^a_v\mu }{\gamma r_a} & 0 \end{bmatrix}
	\end{equation}

	And the $R_0$ is the eigenvalue of maximum norm,

	\begin{equation}
		R_0= \sqrt{\frac{\beta^y_p N_p}{\gamma}}\sqrt{\frac{\beta^y_v\mu b_y}{\gamma(b_yr^2_y+r^1_yr^2_y)}}
	\end{equation}

\section{Controlled Model}

	\begin{align}
		\dot{S^y_p} &= -\beta^y_p S^y_p I_v + (r^1_y + u_1(t)) L^y_p + (r^2_y+u_2(t)) I^y_p + (r_a+u_3(t)) I^a_p-\alpha S^y_p\\
		\dot{S^a_p} &= -\beta^a_p S^a_p I_v + \alpha S^y_p\\
		\dot{L^y_p} &= \beta^y_p S^y_p I_v - b_y L^y_p - (r^1_y+u_1(t)) L^y_p\\
		\dot{L^a_p} &= \beta^a_p S^a_p I_v -b_a L^a_p\\
		\dot{I^y_p} &= b_y L^J_p - (r^2_y+u_2(t)) I^y_p\\
		\dot{I^a_p} &= b_a L^a_p - (r_a+u_3(t)) I^a_p\\
		\dot{S_v} &= -\beta^y_v S_v I^y_p - \beta^a_v S_v I^a_p -(\gamma+u_4(t)) S_v +(1-\theta) \mu \\
		\dot{I_v} &= \beta^y_v S_v I^y_p + \beta^a_v S_v I^a_p - (\gamma+u_4(t)) I_v +\theta \mu
	\end{align}
	
	We need minimize the following functional cost
	
	\begin{equation}
		\begin{aligned}
			J(x,u) &= \int^T_0(A_1L^y_p+A_2I^y_p+A_3 I^a_p+A_4I_v+c_1u_1^2(t)\\
			&+ c_2u_2^2(t)+c_3u_3^2(t)+c_4u_4^2(t))dt
		\end{aligned}
	\end{equation}
	
	By the Pontryagain's principle theorem we have the following adjoint system
	\begin{align}
		\dot{\lambda_1} &= \alpha (\lambda_2 -\lambda_1) + \beta^y_p I_v (\lambda_3 - \lambda_1) \\
		\dot{\lambda_2} &= \beta^a_p I_v (\lambda_4-\lambda_2)\\
		\dot{\lambda_3} &= A_1 + r^1_y (\lambda_1 - \lambda_3) + u_1 ( \lambda_1-\lambda_3)+ b_y (\lambda_5-\lambda_3)\\
		\dot{\lambda_4} &= b_a (\lambda_6 -\lambda_4)\\
		\dot{\lambda_5} &= A_2 + (r^2_y +u_2)(\lambda_1-\lambda_5) + \beta^y_v S_v(\lambda_8-\lambda_7)\\
		\dot{\lambda_6} &= A_3 + (r_a + u_3)(\lambda_1-\lambda_6)+ \beta^a_v S_v (\lambda_8- \lambda_7)\\
		\dot{\lambda_7} &=  (\beta^y_v I^y_p + \beta^a_v I^a_p)(\lambda_8-\lambda_7) - (\gamma+u_4)\lambda_7\\
		\dot{\lambda_8} &= A_4 + \beta^y_p S^y_p (\lambda_3-\lambda_1) + \beta^a_p S^a_p (\lambda_4-\lambda_2)-(\gamma +u_4)\lambda_8
	\end{align}

and the optimallity condition is given by :
	
	\begin{equation}
		\bar{u_1} = \frac{\bar{L^y_p}(\lambda_3-\lambda_1)}{2c_1}
	\end{equation}
	
	\begin{equation}
		\bar{u_2} = \frac{\bar{I^y_p}(\lambda_5 -\lambda_1)}{2c_2}
	\end{equation}
	
	\begin{equation}
		\bar{u_3} = \frac{\bar{I^a_p}(\lambda_6-\lambda_1)}{2c_3}
	\end{equation}
	
	\begin{equation}
		\bar{u_4} = \frac{\bar{S_v}\lambda_7 +\bar{I_v}\lambda_8}{2c_4}
	\end{equation}

%%%%%%%%%%%%%%%%%%%%%%%%%%%%%%%%%%%%%%%%%%%%%%%%%%%%%%%%%%%%%%%%%%%%%%%%%%%%%%%%%%%%%%%%%%%%%%%%%%%%%%%%%%%%%%%%%%%%%%%%%%%%%%%%%%%%%%%%%%%%%%%%%%%%%%%%%%%%%%%%%%%%%%%%%%%%%%%%%%%%%%%%%%%%%%%%%%%%%%%%%%%%%%%%%%%%%%%%%%%%%%%%%%%%%%%%%%%%%%%%%%%%%%%%%%%%%%%%%%%%%%%%%%%%%%%%%%%%%%%%%%%%%%%%%%%%%%%%%%%%%%%%%%%%%%%%%%%%%%%%%%%%%%%%%%%%%%%%%%%%%%%%%%%%
\begin{thebibliography}{9}                                                                                                %
\end{thebibliography}


\appendix

\section{The First Appendix}

\end{document}
